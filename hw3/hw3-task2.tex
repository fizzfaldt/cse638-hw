\documentclass[11pt]{article}

\usepackage{epsfig}
\usepackage{amsfonts}
\usepackage{amssymb}
\usepackage{amstext}
\usepackage{amsmath}
\usepackage{xspace}
\usepackage{theorem}
\usepackage{color}
%\usepackage{times}
\usepackage{graphicx}
\usepackage{url}
\usepackage{enumerate}

%\usepackage{layout}% if you want to see the layout parameters
                     % and now use \layout command in the body

% This is the stuff for normal spacing
\makeatletter
 \setlength{\textwidth}{6.5in}
 \setlength{\oddsidemargin}{0in}
 \setlength{\evensidemargin}{0in}
 \setlength{\topmargin}{0.25in}
 \setlength{\textheight}{8.25in}
 \setlength{\headheight}{0pt}
 \setlength{\headsep}{0pt}
 \setlength{\marginparwidth}{59pt}

 \setlength{\parindent}{0pt}
 \setlength{\parskip}{5pt plus 1pt}
 \setlength{\theorempreskipamount}{5pt plus 1pt}
 \setlength{\theorempostskipamount}{0pt}
 \setlength{\abovedisplayskip}{8pt plus 3pt minus 6pt}

 \renewcommand{\section}{\@startsection{section}{1}{0mm}%
                                   {2ex plus -1ex minus -.2ex}%
                                   {1.3ex plus .2ex}%
                                   {\normalfont\Large\bfseries}}%
 \renewcommand{\subsection}{\@startsection{subsection}{2}{0mm}%
                                     {1ex plus -1ex minus -.2ex}%
                                     {1ex plus .2ex}%
                                     {\normalfont\large\bfseries}}%
 \renewcommand{\subsubsection}{\@startsection{subsubsection}{3}{0mm}%
                                     {1ex plus -1ex minus -.2ex}%
                                     {1ex plus .2ex}%
                                     {\normalfont\normalsize\bfseries}}
 \renewcommand{\paragraph}{\@startsection{paragraph}{4}{0mm}%
                                    {1ex \@plus1ex \@minus.2ex}%
                                    {-1em}%
                                    {\normalfont\normalsize\bfseries}}
 \renewcommand{\subparagraph}{\@startsection{subparagraph}{5}{\parindent}%
                                       {2.0ex \@plus1ex \@minus .2ex}%
                                       {-1em}%
                                      {\normalfont\normalsize\bfseries}}
\makeatother
%\renewcommand{\theenumi}{\Alph{enumi}}

\newenvironment{proof}{{\bf Proof:  }}{\hfill\rule{2mm}{2mm}}
\newenvironment{proofof}[1]{{\bf Proof of #1:  }}{\hfill\rule{2mm}{2mm}}
\newenvironment{proofofnobox}[1]{{\bf#1:  }}{}
\newenvironment{example}{{\bf Example:  }}{\hfill\rule{2mm}{2mm}}
%\renewcommand{\thesection}{\lecnum.\arabic{section}}

\renewcommand{\theequation}{\thesection.\arabic{equation}}
%\renewcommand{\thefigure}{\thesection.\arabic{figure}}
\renewcommand{\thefigure}{\arabic{figure}}

\newtheorem{fact}{Fact}[section]
\newtheorem{lemma}[fact]{Lemma}
\newtheorem{theorem}[fact]{Theorem}
\newtheorem{definition}[fact]{Definition}
\newtheorem{corollary}[fact]{Corollary}
\newtheorem{proposition}[fact]{Proposition}
\newtheorem{claim}[fact]{Claim}
\newtheorem{exercise}[fact]{Exercise}

% math notations
\newcommand{\R}{\ensuremath{\mathbb R}}
\newcommand{\Z}{\ensuremath{\mathbb Z}}
\newcommand{\N}{\ensuremath{\mathbb N}}
\newcommand{\F}{\ensuremath{\mathcal F}}
\newcommand{\SymGrp}{\ensuremath{\mathfrak S}}

\newcommand{\size}[1]{\ensuremath{\left|#1\right|}}
\newcommand{\ceil}[1]{\ensuremath{\left\lceil#1\right\rceil}}
\newcommand{\floor}[1]{\ensuremath{\left\lfloor#1\right\rfloor}}
\newcommand{\poly}{\operatorname{poly}}
\newcommand{\polylog}{\operatorname{polylog}}

% asymptotic notations
\newcommand{\Oh}[1]{\ensuremath{{\mathcal O}\left({#1}\right)}}
\newcommand{\LOh}[1]{\ensuremath{{\mathcal O}\left({#1}\right.}}
\newcommand{\ROh}[1]{\ensuremath{{\mathcal O}\left.{#1}\right)}}
\newcommand{\oh}[1]{\ensuremath{{o}\left({#1}\right)}}
\newcommand{\Om}[1]{\ensuremath{{\Omega}\left({#1}\right)}}
\newcommand{\om}[1]{\ensuremath{{\omega}\left({#1}\right)}}
\newcommand{\Th}[1]{\ensuremath{{\Theta}\left({#1}\right)}}


% pseudocode notations
\newcommand{\xif}{{\bf{\em{if~}}}}
\newcommand{\xthen}{{\bf{\em{then~}}}}
\newcommand{\xelse}{{\bf{\em{else~}}}}
\newcommand{\xelseif}{{\bf{\em{elif~}}}}
\newcommand{\xfi}{{\bf{\em{fi~}}}}
\newcommand{\xendif}{{\bf{\em{endif~}}}}
\newcommand{\xcase}{{\bf{\em{case~}}}}
\newcommand{\xendcase}{{\bf{\em{endcase~}}}}
\newcommand{\xbreak}{{\bf{\em{break~}}}}
\newcommand{\xfor}{{\bf{\em{for~}}}}
\newcommand{\xto}{{\bf{\em{to~}}}}
\newcommand{\xby}{{\bf{\em{by~}}}}
\newcommand{\xdownto}{{\bf{\em{downto~}}}}
\newcommand{\xdo}{{\bf{\em{do~}}}}
\newcommand{\xrof}{{\bf{\em{rof~}}}}
\newcommand{\xwhile}{{\bf{\em{while~}}}}
\newcommand{\xendwhile}{{\bf{\em{endwhile~}}}}
\newcommand{\xand}{{\bf{\em{and~}}}}
\newcommand{\xor}{{\bf{\em{or~}}}}
\newcommand{\xerror}{{\bf{\em{error~}}}}
\newcommand{\xreturn}{{\bf{\em{return~}}}}
\newcommand{\xparallel}{{\bf{\em{parallel~}}}}
\newcommand{\xspawn}{{\bf{\em{spawn~}}}}
\newcommand{\xsync}{{\bf{\em{sync~}}}}
\newcommand{\xarray}{{\bf{\em{array~}}}}
\newcommand{\T}{\hspace{0.5cm}}
\newcommand{\m}{\mathcal}

\def\sland{~\land~}
\def\slor{~\lor~}
\def\sRightarrow{~\Rightarrow~}

\definecolor{gray}{rgb}{0.3,0.3,0.3}

\def\comment#1{\hfill{\color{gray}{$\left\{\textrm{{\em{#1}}}\right\}$}}}
\def\lcomment#1{\hfill{\color{gray}{$\left\{\textrm{{\em{#1}}}\right.$}}}
\def\rcomment#1{\hfill{\color{gray}{$\left.\textrm{{\em{#1}}}\right\}$}}}
\def\fcomment#1{\hfill{\color{gray}{$\textrm{{\em{#1}}}$}}}
\def\func#1{\textrm{\bf{\sc{#1}}}}
\def\mfunc#1{\mbox{\func{#1}}}
\def\funcbf#1{\textrm{\textbf{\textsc{#1}}}}

\newcommand{\hide}[1]{}

\newcommand{\prob}[1]{\ensuremath{\text{{\bf Pr}$\left[#1\right]$}}}
\newcommand{\expct}[1]{\ensuremath{\text{{\bf E}$\left[#1\right]$}}}
\newcommand{\Event}{{\mathcal E}}

\newcommand{\mnote}[1]{\normalmarginpar \marginpar{\tiny #1}}

\makeatletter
   \newcommand\figcaption{\def\@captype{figure}\caption}
   \newcommand\tabcaption{\def\@captype{table}\caption}
\makeatother

 \def\para#1{\vspace{0.2cm}\noindent{\bf{#1.}}}

%%%%%%%%%%%%%%%%%%%%%%%%%%%%%%%%%%%%%%%%%%%%%%%%%%%%%%%%%%%%%%%%%%%%%%%%%%%
% Document begins here %%%%%%%%%%%%%%%%%%%%%%%%%%%%%%%%%%%%%%%%%%%%%%%%%%%%
%%%%%%%%%%%%%%%%%%%%%%%%%%%%%%%%%%%%%%%%%%%%%%%%%%%%%%%%%%%%%%%%%%%%%%%%%%%

\newcommand{\headings}[3]{{\large
{\bf CSE638: Advanced Algorithms, Spring 2013} \hfill {{\bf Date:} #1}\\
\rule[0.01in]{\textwidth}{0.025in}}
%\hline
\begin{center} {{\bf\huge{Homework \##2}}\\{\bf( Due: #3 )}} \end{center}
%\thispagestyle{empty}
}

\begin{document}

\headings{April 23}{3}{May 10}
\newcommand{\lecnum}{5}  % Lecture Number


\para{Task 2} {\bf{[ 70 Points ]}} {\bf Protein Accordian Folding.}

\begin{enumerate}[(a)]
    \item Show that a straightforward iterative implementation of the recurrence for
        \func{Score} runs in $\Oh{n^4}$ time and uses $\Oh{n^2}$ space.

        \bigskip

        See \func{Optimal-Score-A}.
        \func{Score-One-Fold-A} takes $\Oh{n}$ time to run and is run $\Oh{n^3}$ times.
        It uses $\Oh{n^2}$ space for the \func{Score}$(~i,~j~)$ matrix.

    \item Explain how to modify the implementation from part 2(a) to run in $\Oh{n^3}$ time at the cost of increasing
        the space complexity to $\Oh{n^3}$.  Analyze its cache-complexity.

        \bigskip

        Modification:
        See \func{Optimal-Score-B}.  Instead of calculating \func{Score-One-Fold} on the fly, it generates all $\Oh{n^3}$ combinations in advance, taking $\Oh{n^3}$ time and space.  Then lookup is $\Oh{1}$ and so the total time and space is $\Oh{n^3}$ and $\Oh{n^3}$ respectively.
        Both \func{Optimal-Score-B} and \func{Preprocess-Score-One-Fold-B} run in $\Oh{n^3}$ time.

        \bigskip
        Cache-complexity: {\bf \em TODO (what is here is just notes)}
        \func{Preprocess-Score-One-Fold-B}:
        For each $1\leq j \leq n$ we do $n$ scans over $P$.
        We can assume $M > n$ so this is a \func{Scan}  ($\Oh{n/B}$ transfers).

        Also we read and write to the $3$d array \func{Score-One-Fold-B}.
        We need at least $\Oh{n^3/B}$ transfers (because we use $\Oh{n^3}$ space).
        \func{Preprocess-Score-One-Fold-B} can match that 

    \item Show that the space usage of the implementation from part 2(b) can be reduced to
        $\Oh{n^2}$ without increasing its running time. Analyze its cache-complexity.

        \bigskip

        Modification:
        See \func{Optimal-Score-C} and \func{Preprocess-Score-One-Fold-C}.
        There is redundancy in \func{Score-One-Fold} such that the only parameters that matter are $j$ and $\min{(~j-i, ~k-j-1~)}$.
        
        \func{Optimal-Score-C} runs in $\Oh{n^3}$ time and \func{Preprocess-Score-One-Fold-B} run in $\Oh{n^2}$ time.
        Both use $\Oh{n^2}$ memory now.

        \bigskip
        Cache-complexity: {\bf \em TODO (what is here is just notes)}

    \item Convert your iterative algorithm from part 2(c) into a recursive divide-and-conquer algorithm
        that runs in $\Oh{n^3}$ time, uses $\Oh{n^2}$ space, and incurs only $\Oh{\frac{n^3}{B\sqrt{M}}}$ cache-misses,
        where $M$ is the size of the cache and $B$ is the block transfer size.  Explain why the algorithm from part 2(b)
        cannot be converted in the same way to achieve the same asymptotic cache-complexity.

        \bigskip
        {\bf \em TODO (what is here is just notes)}

    \item Find some invariants and modify algorithm from part 2(c) to run in $\Oh{n^2}$ time and use $\Oh{n^2}$ space.
        See \func{Optimal-Score-E}

\end{enumerate}

%%%%%%%%%%%% Begin A
\begin{figure*}[p!]
% \vspace{0.8cm}
    \begin{minipage}{\textwidth}
        \begin{center}

            \framebox{
                \begin{minipage}{\textwidth}
                    {\footnotesize
                        \medskip\noindent\func{Optimal-Score-A}$(~P ~)$

                        \vspace{0.1cm}
                        \noindent
                        ({\color{gray} $P[1:n]$ is a sequence of amino acids.})

                        \noindent
                        \begin{enumerate}[1.]
                            \item \xfor $i \gets n$ \xdownto $1$ \xdo
                            \item \T \xfor $j \gets n$ \xdownto $i+1$ \xdo
                            \item \T \T \xif $j \geq n-1$ \xthen
                                \comment{\func{Score} is undefined without this change.  Original: \xif $j = n$.}
                            \item \T \T \T \func{Score}$(~i,~j~) \gets 0$
                            \item \T \T \xelse
                            \item \T \T \T $best \gets 0$
                            \item \T \T \T \xfor $k \gets n$ \xdownto $j+2$
                            \item \T \T \T \T $candidate \gets$ \func{Score-One-Fold-A}$(~i,~j,~k~) +$ \func{Score}$(~j+1,~k~)$
                            \item \T \T \T \T $best \gets \max{(~candidate, ~best~)}$
                            \item \T \T \T \func{Score}$(~i,~j~) \gets best$
                            \item $answer \gets 0$
                            \item \xfor $j \gets 1$ \xto $n$ \xdo
                            \item \T $answer \gets \max{(~answer, ~\mfunc{Score}(~1, ~j~))}$
                            \item \xreturn $answer$
                        \end{enumerate}

                    }
                \end{minipage}
                \vspace{0.1cm}
            }

            \framebox{
                \begin{minipage}{\textwidth}
                    {\footnotesize
                        \medskip\noindent\func{Score-One-Fold-A}$(~i, ~j, ~k, ~P ~)$

                        \vspace{0.1cm}
                        \noindent
                        ({\color{gray} $P[1:n]$ is a sequence of amino acids, where $n > k-1 > j > i > 0$.  This function counts the number of aligned hydrophobic pairs when the segment $P[i:k]$ is folded at locations $j$ and $j+1$.})

                        \noindent
                        \begin{enumerate}[1.]
                            \item $c \gets 0$

                            \item \xfor $\ell \gets 1$ \xto $\min{(~j-i, ~k-j-1~)}$ \xdo
                            \item \T \xif \func{Hydrophobic}$(~P[j-\ell]~)$ \xand \func{Hydrophobic}$(~P[j+1+\ell]~)$ \xthen
                            \item \T \T $c \gets c + 1$
                            \item \xreturn $c$
                        \end{enumerate}

                    }





                \end{minipage}
                \vspace{0.1cm}
            }
        \end{center}
    \end{minipage}
\end{figure*}

%%%%%%%%%%%% Begin B
\begin{figure*}[p!]
% \vspace{0.8cm}
    \begin{minipage}{\textwidth}
        \begin{center}

            \framebox{
                \begin{minipage}{\textwidth}
                    {\footnotesize
                        \medskip\noindent\func{Optimal-Score-B}$(~P ~)$

                        \vspace{0.1cm}
                        \noindent
                        ({\color{gray} $P[1:n]$ is a sequence of amino acids.})

                        \noindent
                        \begin{enumerate}[1.]
                            \item \func{Score-One-Fold-B} $\gets$ \func{Preprocess-Score-One-Fold-B}$(~P~)$
                            \item \xfor $i \gets n$ \xdownto $1$ \xdo
                            \item \T \xfor $j \gets n$ \xdownto $i+1$ \xdo
                            \item \T \T \xif $j \geq n-1$ \xthen
                            \item \T \T \T \func{Score}$(~i,~j~) \gets 0$
                            \item \T \T \xelse
                            \item \T \T \T $best \gets 0$
                            \item \T \T \T \xfor $k \gets n$ \xdownto $j+2$
                            \item \T \T \T \T $candidate \gets$ \func{Score-One-Fold-B}$(i,j,k) +$ \func{Score}$(j+1,k)$
                            \item \T \T \T \T $best \gets \max{(candidate, best)}$
                            \item \T \T \T \func{Score}$(i,j) \gets best$
                            \item $answer \gets 0$
                            \item \xfor $j \gets 1$ \xto $n$ \xdo
                            \item \T $answer \gets \max{(~answer, ~\mfunc{Score}(~1, ~j~))}$
                            \item \xreturn $answer$
                        \end{enumerate}

                    }
                \end{minipage}
                \vspace{0.1cm}
            }
            \framebox{
                \begin{minipage}{\textwidth}
                    {\footnotesize
                        \medskip\noindent\func{Preprocess-Score-One-Fold-B}$(~P ~)$

                        \vspace{0.1cm}
                        \noindent
                        ({\color{gray} $P[1:n]$ is a sequence of amino acids. Returns a $3$d memoized array for \func{Score-One-Fold-A}, ($0$ when \func{Score-One-Fold-A} is undefined).}

                        \noindent
                        \begin{enumerate}[1.]
                            \item \xfor $j \gets 1$ \xto $n$ \xdo
                            \item \T \xfor $i \gets j$ \xdownto $1$ \xdo
                            \item \T \T \xfor $k \gets j$ \xto $n$ \xdo
                            \item \T \T \T $\ell \gets \min{(~j-i, ~k-j-1~)}$
                            \item \T \T \T \xif $\ell < 1$ \xthen
                            \item \T \T \T \T \func{Score-One-Fold-B}$(~i, ~j, ~k ~) \gets 0$
                            \item \T \T \T \xelseif \func{Hydrophobic}$(P[j-\ell])$ \xand \func{Hydrophobic}$(P[j+1+\ell])$ \xthen
                            \item \T \T \T \T \func{Score-One-Fold-B}$(~i, ~j, ~k ~) \gets$ \func{Score-One-Fold-B}$(~j-\ell+1, ~j, ~j+\ell ~) + 1$
                            \item \T \T \T \xelse
                            \item \T \T \T \T \func{Score-One-Fold-B}$(~i, ~j, ~k ~) \gets$ \func{Score-One-Fold-B}$(~j-\ell+1, ~j, ~j+\ell ~)$
                            \item \xreturn \func{Score-One-Fold-B}
                        \end{enumerate}

                    }





                \end{minipage}
                \vspace{0.1cm}
            }

        \end{center}
    \end{minipage}
\end{figure*}

%%%%%%%%%%%% Begin C
\begin{figure*}[p!]
% \vspace{0.8cm}
    \begin{minipage}{\textwidth}
        \begin{center}

            \framebox{
                \begin{minipage}{\textwidth}
                    {\footnotesize
                        \medskip\noindent\func{Optimal-Score-C}$(~P ~)$

                        \vspace{0.1cm}
                        \noindent
                        ({\color{gray} $P[1:n]$ is a sequence of amino acids.})

                        \noindent
                        \begin{enumerate}[1.]
                            \item \func{Score-One-Fold-C} $\gets$ \func{Preprocess-Score-One-Fold-C}$(~P~)$
                            \item \xfor $i \gets n$ \xdownto $1$ \xdo
                            \item \T \xfor $j \gets n$ \xdownto $i+1$ \xdo
                            \item \T \T \xif $j \geq n-1$ \xthen
                            \item \T \T \T \func{Score}$(~i,~j~) \gets 0$
                            \item \T \T \xelse
                            \item \T \T \T $best \gets 0$
                            \item \T \T \T \xfor $k \gets n$ \xdownto $j+2$
                            \item \T \T \T \T $\ell \gets \min{(~j-i, ~k-j-1~)}$
                            \item \T \T \T \T $candidate \gets$ \func{Score-One-Fold-C}$(j,\ell) +$ \func{Score}$(j+1,k)$
                            \item \T \T \T \T $best \gets \max{(candidate, best)}$
                            \item \T \T \T \func{Score}$(i,j) \gets best$
                            \item $answer \gets 0$
                            \item \xfor $j \gets 1$ \xto $n$ \xdo
                            \item \T $answer \gets \max{(~answer, ~\mfunc{Score}(~1, ~j~))}$
                            \item \xreturn $answer$
                        \end{enumerate}

                    }
                \end{minipage}
                \vspace{0.1cm}
            }
            \framebox{
                \begin{minipage}{\textwidth}
                    {\footnotesize
                        \medskip\noindent\func{Preprocess-Score-One-Fold-C}$(~P ~)$

                        \vspace{0.1cm}
                        \noindent
                        ({\color{gray} $P[1:n]$ is a sequence of amino acids. Returns a $2$d subset of \func{Score-One-Fold-B}, where we only have the answer for each $1\leq j,\ell \leq n$.})

                        \noindent
                        \begin{enumerate}[1.]
                            \item \xfor $j \gets 1$ \xto $n$ \xdo
                            \item \T \xfor $\ell \gets 0$ \xto $\min{(~j-1, ~n-j-1~)}$ \xdo
                            \item \T \T \xif $\ell < 1$ \xthen
                            \item \T \T \T \func{Score-One-Fold-C}$(~j, ~\ell~) \gets 0$
                            \item \T \T \xelseif \func{Hydrophobic}$(P[j-\ell])$ \xand \func{Hydrophobic}$(P[j+1+\ell])$ \xthen
                            \item \T \T \T \func{Score-One-Fold-C}$(~j, ~\ell~) \gets$ \func{Score-One-Fold-C}$(~j, ~\ell-1~) + 1$
                            \item \T \T \xelse
                            \item \T \T \T \func{Score-One-Fold-C}$(~j, ~\ell~) \gets$ \func{Score-One-Fold-C}$(~j, ~\ell-1~)$
                            \item \xreturn \func{Score-One-Fold-B}
                        \end{enumerate}

                    }





                \end{minipage}
                \vspace{0.1cm}
            }

        \end{center}
    \end{minipage}
\end{figure*}

%%%%%%%%%%%% Begin E
\begin{figure*}[p!]
% \vspace{0.8cm}
    \begin{minipage}{\textwidth}
        \begin{center}

            \framebox{
                \begin{minipage}{\textwidth}
                    {\footnotesize
                        \medskip\noindent\func{Optimal-Score-E}$(~P ~)$

                        \vspace{0.1cm}
                        \noindent
                        ({\color{gray} $P[1:n]$ is a sequence of amino acids.})

                        \noindent
                        \begin{enumerate}[1.]
                            \item \xfor $j \gets 1$ \xto $n$ \xdo
                            \item \T $\mfunc{Initialized}(~j~) \gets \mfunc{False}$
                            \item \T \xfor $k \gets 1$ \xto $n$ \xdo
                            \item \T \T \mfunc{Best-Score}$(~j, ~k~) \gets 0$
                                \comment{\mfunc{Best-Score}$(~j, ~k~) = \max_{k \leq \ell \leq n}(\mfunc{Score}(~j,~\ell~)$}
                            \item \T \T \mfunc{Score}$(~j, ~k~) \gets 0$
                            \item \mfunc{Score-One-Fold-C} $\gets$ \mfunc{Preprocess-Score-One-Fold-C}$(~P~)$
                            \item \xfor $i \gets n$ \xdownto $1$ \xdo
                            \item \T \xfor $j \gets n$ \xdownto $i+1$ \xdo
                            \item \T \T \xif $j < n-1$ \xthen
                            \item \T \T \T \xif $\lnot \mfunc{Initialized}(~j~)$ \xthen
                            \item \T \T \T \T $\mfunc{Initialized}(~j~) \gets \mfunc{True}$
                            \item \T \T \T \T $\mfunc{Best-Score}(~j+1, ~n~) \gets \mfunc{Score}(~j+1, ~n~)$
                            \item \T \T \T \T \xfor $k \gets n-1$ \xdownto $j+1$ \xdo
                            \item \T \T \T \T \T $\mfunc{Best-Score}(~j+1, ~k ~) \gets \max{(~\mfunc{Best-Score}(~j+1,~k+1~),~\mfunc{Score}(~j+1,~k~)~)}$
                            \item \T \T \T $k \gets 2j-i+1$
                                \comment{$j-i = k-j-1$}
                            \item \T \T \T $\ell \gets j-i$
                            \item \T \T \T $C_1 \gets \mfunc{Score}(~i+1,~j~)$
                            \item \T \T \T $C_2 \gets \mfunc{Best-Score}(~j+1,~k~) + \mfunc{Score-One-Fold-C}(~j,~\ell~)$
                            \item \T \T \T $\mfunc{Score}(~i,~j~) \gets \max{(~C_1,~C_2~)}$
                            \item $answer \gets 0$
                            \item \xfor $j \gets 1$ \xto $n$ \xdo
                            \item \T $answer \gets \max{(~answer, ~\mfunc{Score}(~1, ~j~))}$
                            \item \xreturn $answer$
                        \end{enumerate}

                    }
                \end{minipage}
                \vspace{0.1cm}
            }

        \end{center}
    \end{minipage}
\end{figure*}



\end{document}



