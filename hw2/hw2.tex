\documentclass[11pt]{article}

\usepackage{epsfig}
\usepackage{amsfonts}
\usepackage{amssymb}
\usepackage{amstext}
\usepackage{amsmath}
\usepackage{xspace}
\usepackage{theorem}
\usepackage{color}
%\usepackage{times}
\usepackage{graphicx}
\usepackage{url}

%\usepackage{layout}% if you want to see the layout parameters
                     % and now use \layout command in the body

% This is the stuff for normal spacing
\makeatletter
 \setlength{\textwidth}{6.5in}
 \setlength{\oddsidemargin}{0in}
 \setlength{\evensidemargin}{0in}
 \setlength{\topmargin}{0.25in}
 \setlength{\textheight}{8.25in}
 \setlength{\headheight}{0pt}
 \setlength{\headsep}{0pt}
 \setlength{\marginparwidth}{59pt}

 \setlength{\parindent}{0pt}
 \setlength{\parskip}{5pt plus 1pt}
 \setlength{\theorempreskipamount}{5pt plus 1pt}
 \setlength{\theorempostskipamount}{0pt}
 \setlength{\abovedisplayskip}{8pt plus 3pt minus 6pt}

 \renewcommand{\section}{\@startsection{section}{1}{0mm}%
                                   {2ex plus -1ex minus -.2ex}%
                                   {1.3ex plus .2ex}%
                                   {\normalfont\Large\bfseries}}%
 \renewcommand{\subsection}{\@startsection{subsection}{2}{0mm}%
                                     {1ex plus -1ex minus -.2ex}%
                                     {1ex plus .2ex}%
                                     {\normalfont\large\bfseries}}%
 \renewcommand{\subsubsection}{\@startsection{subsubsection}{3}{0mm}%
                                     {1ex plus -1ex minus -.2ex}%
                                     {1ex plus .2ex}%
                                     {\normalfont\normalsize\bfseries}}
 \renewcommand{\paragraph}{\@startsection{paragraph}{4}{0mm}%
                                    {1ex \@plus1ex \@minus.2ex}%
                                    {-1em}%
                                    {\normalfont\normalsize\bfseries}}
 \renewcommand{\subparagraph}{\@startsection{subparagraph}{5}{\parindent}%
                                       {2.0ex \@plus1ex \@minus .2ex}%
                                       {-1em}%
                                      {\normalfont\normalsize\bfseries}}
\makeatother

\newenvironment{proof}{{\bf Proof:  }}{\hfill\rule{2mm}{2mm}}
\newenvironment{proofof}[1]{{\bf Proof of #1:  }}{\hfill\rule{2mm}{2mm}}
\newenvironment{proofofnobox}[1]{{\bf#1:  }}{}
\newenvironment{example}{{\bf Example:  }}{\hfill\rule{2mm}{2mm}}
%\renewcommand{\thesection}{\lecnum.\arabic{section}}

\renewcommand{\theequation}{\thesection.\arabic{equation}}
%\renewcommand{\thefigure}{\thesection.\arabic{figure}}
\renewcommand{\thefigure}{\arabic{figure}}

\newtheorem{fact}{Fact}[section]
\newtheorem{lemma}[fact]{Lemma}
\newtheorem{theorem}[fact]{Theorem}
\newtheorem{definition}[fact]{Definition}
\newtheorem{corollary}[fact]{Corollary}
\newtheorem{proposition}[fact]{Proposition}
\newtheorem{claim}[fact]{Claim}
\newtheorem{exercise}[fact]{Exercise}

% math notations
\newcommand{\R}{\ensuremath{\mathbb R}}
\newcommand{\Z}{\ensuremath{\mathbb Z}}
\newcommand{\N}{\ensuremath{\mathbb N}}
\newcommand{\F}{\ensuremath{\mathcal F}}
\newcommand{\SymGrp}{\ensuremath{\mathfrak S}}

\newcommand{\size}[1]{\ensuremath{\left|#1\right|}}
\newcommand{\ceil}[1]{\ensuremath{\left\lceil#1\right\rceil}}
\newcommand{\floor}[1]{\ensuremath{\left\lfloor#1\right\rfloor}}
\newcommand{\poly}{\operatorname{poly}}
\newcommand{\polylog}{\operatorname{polylog}}

% asymptotic notations
\newcommand{\Oh}[1]{{\mathcal O}\left({#1}\right)}
\newcommand{\LOh}[1]{{\mathcal O}\left({#1}\right.}
\newcommand{\ROh}[1]{{\mathcal O}\left.{#1}\right)}
\newcommand{\oh}[1]{{o}\left({#1}\right)}
\newcommand{\Om}[1]{{\Omega}\left({#1}\right)}
\newcommand{\om}[1]{{\omega}\left({#1}\right)}
\newcommand{\Th}[1]{{\Theta}\left({#1}\right)}


% pseudocode notations
\newcommand{\xif}{{\bf{\em{if~}}}}
\newcommand{\xthen}{{\bf{\em{then~}}}}
\newcommand{\xelse}{{\bf{\em{else~}}}}
\newcommand{\xelseif}{{\bf{\em{elif~}}}}
\newcommand{\xfi}{{\bf{\em{fi~}}}}
\newcommand{\xendif}{{\bf{\em{endif~}}}}
\newcommand{\xcase}{{\bf{\em{case~}}}}
\newcommand{\xendcase}{{\bf{\em{endcase~}}}}
\newcommand{\xbreak}{{\bf{\em{break~}}}}
\newcommand{\xfor}{{\bf{\em{for~}}}}
\newcommand{\xto}{{\bf{\em{to~}}}}
\newcommand{\xby}{{\bf{\em{by~}}}}
\newcommand{\xdownto}{{\bf{\em{downto~}}}}
\newcommand{\xdo}{{\bf{\em{do~}}}}
\newcommand{\xrof}{{\bf{\em{rof~}}}}
\newcommand{\xwhile}{{\bf{\em{while~}}}}
\newcommand{\xendwhile}{{\bf{\em{endwhile~}}}}
\newcommand{\xand}{{\bf{\em{and~}}}}
\newcommand{\xor}{{\bf{\em{or~}}}}
\newcommand{\xerror}{{\bf{\em{error~}}}}
\newcommand{\xreturn}{{\bf{\em{return~}}}}
\newcommand{\xparallel}{{\bf{\em{parallel~}}}}
\newcommand{\xspawn}{{\bf{\em{spawn~}}}}
\newcommand{\xsync}{{\bf{\em{sync~}}}}
\newcommand{\xarray}{{\bf{\em{array~}}}}
\newcommand{\T}{\hspace{0.5cm}}
\newcommand{\m}{\mathcal}

\def\sland{~\land~}
\def\slor{~\lor~}
\def\sRightarrow{~\Rightarrow~}

\definecolor{gray}{rgb}{0.3,0.3,0.3}

\def\comment#1{\hfill{\color{gray}{$\left\{\textrm{{\em{#1}}}\right\}$}}}
\def\lcomment#1{\hfill{\color{gray}{$\left\{\textrm{{\em{#1}}}\right.$}}}
\def\rcomment#1{\hfill{\color{gray}{$\left.\textrm{{\em{#1}}}\right\}$}}}
\def\fcomment#1{\hfill{\color{gray}{$\textrm{{\em{#1}}}$}}}
\def\func#1{\textrm{\bf{\sc{#1}}}}
\def\funcbf#1{\textrm{\textbf{\textsc{#1}}}}

\newcommand{\hide}[1]{}

\newcommand{\prob}[1]{\ensuremath{\text{{\bf Pr}$\left[#1\right]$}}}
\newcommand{\expct}[1]{\ensuremath{\text{{\bf E}$\left[#1\right]$}}}
\newcommand{\Event}{{\mathcal E}}

\newcommand{\mnote}[1]{\normalmarginpar \marginpar{\tiny #1}}

\makeatletter
   \newcommand\figcaption{\def\@captype{figure}\caption}
   \newcommand\tabcaption{\def\@captype{table}\caption}
\makeatother

 \def\para#1{\vspace{0.2cm}\noindent{\bf{#1.}}}

%%%%%%%%%%%%%%%%%%%%%%%%%%%%%%%%%%%%%%%%%%%%%%%%%%%%%%%%%%%%%%%%%%%%%%%%%%%
% Document begins here %%%%%%%%%%%%%%%%%%%%%%%%%%%%%%%%%%%%%%%%%%%%%%%%%%%%
%%%%%%%%%%%%%%%%%%%%%%%%%%%%%%%%%%%%%%%%%%%%%%%%%%%%%%%%%%%%%%%%%%%%%%%%%%%

\newcommand{\headings}[3]{{\large
{\bf CSE638: Advanced Algorithms, Spring 2013} \hfill {{\bf Date:} #1}\\
\rule[0.01in]{\textwidth}{0.025in}}
%\hline
\begin{center} {{\bf\huge{Homework \##2}}\\{\bf( Due: #3 )}} \end{center}
%\thispagestyle{empty}
}

\begin{document}

\headings{March 26}{2}{April 16}
\newcommand{\lecnum}{5}  % Lecture Number




\para{Task 1} {\bf{[ 40 Points ]}} {\bf Parallel BFS with Cilk's Work Stealing.}
%
 \begin{itemize}

 \item[$(a)$] {\bf{[ 20 Points ]}} In homework 1 we analyzed and implemented
parallel BFS algorithms based on randomized distributed work-stealing,
and we explicitly implemented the work-stealing mechanism. Explain how
you will reimplement the algorithm from part $1(f)$ of homework 1
(without the modication for avoiding duplicate exploration) so
that it uses Cilk's work-stealing scheduler for load balancing
instead of your own work-stealing code. 

 \item[$(b)$] {\bf{[ 20 Points ]}} Update the table you created
for part $1(j)$ of homework 1 with the running times of your
new implementation.

\begin {table}[H]
\begin{center}
    \begin{tabular}{ | c | c | c | c | c | c | c | c | }
    \hline
    Input File & $RT_{SBFS}$ & $RT_{PBFS^{(f)}}$ & $RT_{PBFS^{(g)}}$ & $SF_{PBFS^{(f)}}$ & $SF_{PBFS^{(g)}}$ & $T_{HW2}$ & $S_{HW2}$\\ \hline
    cage14 & 1786 & 313 & 315.18 & 5.70 & 5.67 & 55.65 & 1 \\ \hline
    cage15 & 1231 & 303.904 & 303.75 & 4.05 & 4.05 & 212.18 & 1 \\ \hline
    freescale & 1835 & 223 & 253.00 & 8.22 & 7.25 & 218.56 & 1 \\ \hline
	Wikipedia & 709 & 705.47 & 703 & 1.00 & 1.00 & 139.19 & 1 \\ \hline
    kkt-power & 410 & 21.5 & 20.79 & 19.07 & 19.72 & 10.06 & 1 \\ \hline
    RMAT100M & 3253 & 673.57 & 647.00 & 4.83 & 5.03 & 501.59 & 1 \\ \hline
    RMAT1B & - & 2544.57 & - & - & - & 1 & 1 \\ \hline
        
    \end{tabular}
    \caption {RT = Running Time, SBFS = Serial BFS, $PBFS^{(f)}$ = Parallel BFS for task 1(f), $PBFS^{(g)}$ = Parallel BFS for task 1(g) and SF = Speed Factor. Times are in seconds.}
\end{center}
\end {table}


 \end{itemize}


\para{Task 2} {\bf{[ 160 Points ]}} {\bf Parallel Connected Components.}
%
 \begin{itemize}

 \item[$(a)$] {\bf{[ 10 Points ]}}
 Consider the function \func{Random-Hook}$(~V, ~L, ~N ~)$ in Figure \ref{fi:pcc},
 and assume that $V$ has no zero degree vertices.
 Suppose we start with an empty set $Q$, and add edges to it as follows.
 We traverse the vertices in $V$ in some order, and for each
 $v \in V$ encountered in that order, we add  $\left( v, N[ v ] \right)$ 
 to $Q$ provided the graph induced
 by $Q \cup \left\{ \left( v, N[ v ] \right) \right\}$ does not
 contain a cycle. Prove that $|Q| \geq \frac{|V|}{2}$.


For an arbitrary graph that satisfies the required property let us assume for the sake of contradiction, the largest possible $Q$ is such that $|Q| < \frac{|V|}{2}$. $G' = (V, Q')$ is cycle free. Adding more edges will lead to the formation of a cycle in $G'$.\\
\\
Consider all the potential edges that can be added to graph $E' = \left\{ \left( v, N[ v ] \right), v \in V \right\}$. $ |E'| = |V|$, since each vertex $u$, has a unique $N[u]$.
Clearly  $|Q| < \frac{|E'|}{2}$.\\
\\
If there are k edges in a graph, at most $2 \times k$ vertices can have an edge incident on them. So $G'$ has less than $|V|$ vertices that have an edge incident on them. There is atleast one non-isloated vertex v, that has no edge incident on it in $G'$.\\
\\
 The partition $V \setminus \left\{v \right\}, \left\{v \right\}$ in $G'$ form two cycle free components. Adding an edge between them does not create a cycle. Thus $\left( v, N[ v ] \right) $ can be added to $Q$ without creating a cycle. This contradicts our assumption that a maximal subset of edges $Q$, can have $|Q| < \frac{|V|}{2}$. Thus, $|Q| \geq \frac{|V|}{2}$.
 \item[$(b)$] {\bf{[ 10 Points ]}} 
 Consider the set $Q$ constructed in part $2(a)$. 
 We say that an edge $\left( u, v \right) \in Q$ is a {\em hook} provided
 $C[ u ] \neq C[ v ]$  right after step 3 of \func{Random-Hook}.
%
  Prove that for all $Q' \subset Q$, 
 if one knows the {\em hook} status of all edges in $Q'$ that
 still does not reveal anything about the hook status
 of any edge in $Q \setminus Q'$.


 \item[$(c)$] {\bf{[ 10 Points ]}} Explain why \func{Random-Hook}
 (in Figure \ref{fi:pcc}) is superior to the random hooking  
 technique we saw in the class (used in \func{Par-Randomized-CC}).
\\
\func{Par-Randomized-CC} can hook vertices to a root vertex such that it forms a star of depth at most $1$.
 Also the expected number of edges thus hooked in a round is exactly $\frac{|E|}{2}$.\\
In \func{Random-Hook} during the second hooking phase, a vertex $u$ hooked to root $v$
 can in turn add a new vertex to the star. Thus the depth of a start can be at most $2$. 
Also the number of vertices hooked in a round is at least $\frac{|V|}{4}$.
\\ 
 \item[$(d)$] {\bf{[ 10 Points ]}} Use your results from parts
 $2(a)$--$2(c)$ to show that in any invocation of \func{Random-Hook}$(~V, ~L, ~N ~)$,
 with probability at least $1 - {1 \over {e^{|V| / 32}}}$
 at least $\frac{|V|}{16}$ vertices of $V$ will change
 their $L$ values. 
\\
Let $X$ be the indicator random variable that represents whether $L$ value of a vertex changes.
For $L$ value of a vertex to change, during the first phase of hooking, $C [v] = \func{HEAD}$ and  $C[ N[v] ] = \func{HEAD}$.\\
 The $L$ value can change also during the second phase of hooking after flipping vertices that are not hooked.\\
So the probability that $L$ value of a vertex changes is at least equal to the probaility that $L$ value changes during the first phase of hooking which is $\frac{1}{4}$.
\\
Thus expected number vertices for which $L$ value changes, $\mu = E[X] = |V| \times \frac{1}{4}$. $ \delta = 3 \over 4$.
\\
Applying, Chernoff Bound's lower tail:
\[
P \left[ X < \frac{|V|}{16} \right] < e^{\frac { \frac{|V|}{4} \times {\frac{3}{4}}^2} {2}} = e^{|V| / 32}
\]

Thus,
$P \left[ X \geq \frac{|V|}{16} \right] \geq 1 - {1 \over {e^{|V| / 32}}}$.



 \item[$(e)$] {\bf{[ 10 Points ]}} Use your result from part $2(d)$
 to prove high probability bounds
 for the work and span of \func{Par-Randomized-CC-2}. 
 
 \item[$(f)$] {\bf{[ 15 Points ]}} Consider \func{Par-Randomized-CC-3}.
 The algorithm repeatedly chooses a random sample (of geometrically decreasing
 sizes) from the edges of the input graph, and uses the sample to identify vertices
 (or supervertices) that have {\em potentially high degree} (PhD).
 Each vertex of the graph starts with a PhD status, but loses
 the status as soon as a chosen sample of edges fails to 
 include an edge connecting that vertex with another vertex
 with a PhD status. The \func{Random-Hook} function is called only on the vertices
 that retain the PhD status. After a sufficient number of such
 sampling and hooking rounds, the number of edges among the
 supervertices reduce to a sufficiently small number. At that
 point connected components among the supervertices are
 found by calling \func{Par-Randomized-CC-2}. Now suppose
 at recursion depth $d$ of \func{Par-Randomized-CC-3},
 $n_d$ denotes the number of vertices/supervertices
 in $V$ with a PhD status. Then $n_0$ is the number 
 of vertices in the original input graph. Let $n = n_0$.
 Prove that for each
 $d \in [0, d_{max}]$, $n_d \leq {\alpha}^{2d} \cdot n$ w.h.p. in $n$, where 
 $\alpha = \sqrt{\frac{15}{16}}$
 and $d_{max} = \ceil{ \frac{1}{4} \log_{\frac{1}{\alpha}}{n} }$.

 \item[$(g)$] {\bf{[ 15 Points ]}} We call an edge $(u, v)$ {\em heavy}
 provided $PhD[ u ] = PhD[ v ] = \func{True}$, otherwise we call it
 {\em light}. At recursion depth $d$ of \func{Par-Randomized-CC-3},
 let $r_d$ be the expected number of heavy edges that become light.
 Prove that for each $d \in [0, d_{max}]$, $r_d \leq {\alpha}^{d} \cdot n$.
\\
The question is wrong. Consider the complete graph as an input. At the beginning all edges and vertices are heavy. The first time you run random hook, the expected number of vertices that lose PhD status is at least $\frac{1}{4} n$. And so in the next level at least  $\frac{1}{4} n (n-1) \not \leq \alpha^d n$ heavy edges become light.
 \\
 \item[$(h)$] {\bf{[ 10 Points ]}} Prove that in the call to
 \func{Par-Randomized-CC-2} in line 21 of \func{Par-Randomized-CC-3},
 the expected number of edges in $E'$ is $\Oh{n}$.

See part(g). Expected number of heavy edges at the start of level $d  \leq {\frac{9}{16}}^d n$
Expected number of heavy edges at the start of level $d_{max}  \leq {\frac{9}{16}}^d_{max} n \leq \alpha^{2d_{max}} n \leq n^{\frac{1}{2}$
\[
\]
\\
 \item[$(i)$] {\bf{[ 10 Points ]}} Compute the expected work
 and span of \func{Par-Randomized-CC-3}.

 \item[$(j)$] {\bf{[ 60 Points ]}} Implement \func{Par-Randomized-CC}
 (the one we saw in the class), \func{Par-Randomized-CC-2} (from Figure \ref{fi:pcc})
 and \func{Par-Randomized-CC-3} (from Figure \ref{fi:pcc-sampling}), and optimize
 them. Create a table that compares the running
 times of these three implementations using all cores. For each input file (in Appendix 1)
 create a separate row in the table showing the running times of all three algorithms.

 \item[$(k)$] {\bf{[ 20 Points ]}} Generate a {\tt Cilkview} strong scalability plot 
 (see slides 15--18 of lecture 5) for each of the two parallel implementations from
 part $(j)$ using the Live Journal graph (see Appendix 1) as input.

 \end{itemize}


\begin{figure*}[p!]
% \vspace{0.8cm}
 \begin{minipage}{\textwidth}
 \begin{center}
 \framebox{
 \begin{minipage}{\textwidth}
 {\footnotesize
 \medskip\noindent\func{Par-Randomized-CC-2}$(~V, ~E, ~L )$

 \vspace{0.1cm}
 \noindent
 ({\color{gray} Input is an unweighted undirected graph with vertex set $V$ and 
 edge set $E$. For each $v \in V$, $L[ v ]$ is set to $v$
 before the invocation of this function. 
 % We assume that $(u, v) \in E \Rightarrow (v, u) \in E$.  
 When this function terminates, for each $v \in V$, $L[ v ]$
 contains the unique id of the connected component containing $v$.
 We call an edge $(u, v)$ {\em live} provided $L[ u ] \neq L[ v ]$.})

 \noindent
 \begin{enumerate}

 \item \xif $|E| = 0$ \xthen \xreturn
         \comment{no edge to contract} 

 \item \xparallel \xfor each $(u, v) \in E$ \xdo $N[ u ] \gets v$,~ $N[ v ] \gets u$
	\comment{try to associate the edge $(u, v)$ with $u$ and $v$}

 \item \func{Random-Hook}$(~V, ~L, ~N ~)$
        \comment{hook among vertices in $V$ based on the edges chosen in the previous step}

 \item $V' \leftarrow \left\{~v~|~v \in V~ \wedge ~v = L[v] ~\right\}$
        \comment{$V'$ contains only the roots after hooking}

 \item $E' \leftarrow \left\{~( L[ u ], L[ v ])~|~(u, v) \in E~ \wedge ~L[u] \neq L[v] ~\right\}$
           \lcomment{$E'$ contains only edges among roots,}

           \rcomment{and no duplicate edges and self loops}

 \item \func{Par-Randomized-CC-2}$(~V', ~E', ~L ~)$
              \comment{recurse on the smaller instance}

 \item \xparallel \xfor each $v \in V$ \xdo $L[ v ] \gets L[ L[ v ] ]$ 
              \comment{map the solution back to the current instance}


 \vspace{0.2cm}

 \end{enumerate}

 }
 \end{minipage}
 \vspace{0.1cm}
 }


 \framebox{
 \begin{minipage}{\textwidth}
 {\footnotesize
 \medskip\noindent\func{Random-Hook}$(~V, ~L, ~N ~)$

 \vspace{0.1cm}
 \noindent
 ({\color{gray} Input is an unweighted undirected graph with vertex set $V$.
 For each $v \in V$, $L[ v ]$ is set to $v$
 before the invocation of this function. 
 For each $u \in V$, $N[ u ]$ is set to a $v$
 such that $(u, v)$ is an edge in the graph.
 This function randomly hooks vertices in $V$ to their neighbors 
 in such a way that after the function terminates these vertices
 form a set of disjoint stars. For each $v \in V$, $L[ v ]$ 
 is set to $u$ (possibly $u = v$) provided $u$ is the center of the star containing $v$.})

 \noindent
 \begin{enumerate}

 \item \xparallel \xfor each $u \in V$ \xdo 
          \comment{for each vertex in $V$} 

 \item \T $C_u \gets \func{Random}\left\{~ \func{Head},~ \func{Tail} ~\right\}$
          \comment{toss a coin} 

 \item \T $H_u \gets \func{False}$
          \comment{record that this vertex has not yet been hooked} 

 \item \xparallel \xfor each $u \in V$ \xdo 
         \comment{for each vertex $u$ in $V$} 

 \item \T $v \gets N[ u ]$
	 \comment{will try to hook $u$ with $v = N[ u ]$}

 \item \T \xif $C_u = \func{Tail}$ \xand $C_{v} = \func{Head}$ \xthen 
         \comment{if $u$ tossed \func{Tail} and $v$ tossed \func{Head}}

 \item \T\T $L[ u ] \gets v$
         \comment{make $u$ point to $v$}

 \item \T\T $H_u \gets \func{True}$,~ $H_{v} \gets \func{True}$
         \comment{record that both $u$ and $v$ are hooked}

 \item \xparallel \xfor each $u \in V$ \xdo 
        \comment{manipulate the coin tosses to hook more in a second try}

 \item \T \xif $H_{u} = \func{True}$ \xthen $C_u \gets \func{Head}$
           \comment{if $u$ is already hooked, will try to hook unhooked vertices pointing to $u$}

 \item \T \xelse \xif $C_u = \func{Tail}$ \xthen $C_u \gets \func{Head}$ \xelse $C_u \gets \func{Tail}$
           \comment{if $u$ is not hooked, flip $C_u$}
 
 \item \xparallel \xfor each $u \in V$ \xdo 
         \comment{try to hook again}

 \item \T $v \gets N[ u ]$
	 \comment{will try to hook $u$ with $v = N[ u ]$}

 \item \T \xif $C_u = \func{Tail}$ \xand $C_{v} = \func{Head}$ \xthen 
         \comment{if $u$ has \func{Tail} and $v$ has \func{Head}}

 \item \T\T $L[ u ] \gets L[ v ]$
         \comment{make $u$ point to whatever $v$ is pointing to}

 \vspace{0.2cm}

 \end{enumerate}

 }
 \end{minipage}
 \vspace{0.1cm}
 }

 \vspace{-0.2cm}
 \caption{Parallel connected components (CC) on a graph.}
 \label{fi:pcc}
 \vspace{-0.3cm}
 \end{center}
 \end{minipage}
 \end{figure*}


\begin{figure*}[p!]
% \vspace{0.8cm}
 \begin{minipage}{\textwidth}
 \begin{center}
 \framebox{
 \begin{minipage}{\textwidth}
 {\footnotesize
 \medskip\noindent\func{Par-Randomized-CC-3}$(~V, ~E, ~L, ~PhD,~ N, ~U, ~d ~)$

 \vspace{0.1cm}
 \noindent
 ({\color{gray} Input is an unweighted undirected graph with vertex set $V$ and 
 edge set $E$.
 The recursion depth of the function is given by $d$ which
 is set to $0$ when the function is invoked for the first time. 
 Let $n$ be the number of vertices in the graph when $d = 0$,
 and $m = |E|$. Each $v \in V$ is an integer in $[1, n]$.
 Pointers $L$ (label), $PhD$ (potentially high degree), $N$ (neighbor) and $U$ (updated) 
 point to arrays $L[1 : n]$, $PhD[1 : n]$, $N[1 : n]$ and $U[1 : n]$,
 respectively. For each $v \in [1, n]$, $L[ v ]$ is set to $v$,
 and $PhD[ v ]$ is set to $\func{True}$ before the initial invocation of this function. 
 When this function terminates, for each $v \in V$, $L[ v ]$
 contains the unique id of the connected component containing $v$.
 We assume that $\alpha = \sqrt{\frac{15}{16}}$ and 
 $d_{max} = \ceil{ \frac{1}{4} \log_{\frac{1}{\alpha}}{n} }$.
 We call an edge $(u, v)$ {\em live} provided $L[ u ] \neq L[ v ]$.
 Edge $(u, v)$ is {\em heavy} provided $PhD[ u ] = PhD[ v ]$,
 otherwise it is {\em light}.})

 \noindent
 \begin{enumerate}

 \item \xif $d \leq d_{max}$ \xthen
           \comment{need to recurse more to sufficiently reduce \#vertices with $PhD$ status}

 \item \T $m_d \gets \ceil{m \cdot \alpha ^ d }$
           \comment{size of edge sample which geometrically decreases with $d$}

 \item \T $\widehat{E} \gets$ a sample of size $m_d$ chosen uniformly at random from $E$
           \comment{do not always touch all edges in $E$}

 \item \T \xparallel \xfor each $v \in V$ \xdo $U[ v ] \gets \func{False}$
           \comment{flag $U[ v ]$ keeps track if an edge in $\widehat{E}$ hits $v$}

 \item \T \xparallel \xfor each $(u, v) \in \widehat{E}$ \xdo 
           \comment{check each edge in the sample}

 \item \T\T $u' \gets L[ u ]$,~ $v' \gets L[ v ]$
           \comment{find the root of the tree containing each endpoint}

 \item \T\T \xif $u' \in V$ \xand $v' \in V$ \xand $u' \neq v'$ \xand $PhD[ u' ] = PhD[ v' ] = \func{True}$ \xthen 
           \comment{if $(u', v')$ is live and heavy}

 \item \T\T\T $N[ u' ] \gets v'$,~ $N[ v' ] \gets u'$
           \comment{try to associate the edge with $u'$ and $v'$}

 \item \T\T\T $U[ u' ] \gets \func{True}$,~ $U[ v' ] \gets \func{True}$
           \comment{$\widehat{E}$ hits $u'$ and $v'$}

 \item \T \xparallel \xfor each $v \in V$ \xdo
              \comment{check each vertex $v$ in $V$}

 \item \T\T \xif $U[ v ] = \func{false}$ \xthen $PhD[ v ] \gets \func{False}$
              \comment{if $\widehat{E}$ does not hit $v$ then $v$ loses its $PhD$ status}

 \item \T $\widehat{V} \gets \left\{~v~|~v \in V~ \wedge ~U[ v ] = \func{True} ~\right\}$
              \comment{$\widehat{V}$ contains the vertices from $V$ which still have $PhD$ status}

 \item \T \func{Random-Hook}$(~\widehat{V}, ~L, ~N ~)$
              \comment{hook among vertices in $\widehat{V}$ (see Figure \ref{fi:pcc})}

 \item \T $V' \gets \left\{~v~|~v \in V~ \wedge ~v = L[v] ~\right\}$
              \comment{$V'$ contains only the roots after hooking}

 \item \T \func{Par-Randomized-CC-3}$(~V', ~E, ~L, ~PhD, ~N, ~U, ~d+1 ~)$
              \comment{recurse on the smaller instance}

 \item \T \xparallel \xfor each $v \in V$ \xdo $L[ v ] \gets L[ L[ v ] ]$
              \comment{map the solution back to the current instance}

 \item \xendif 

 \item \xif $d = 0$ \xthen
         \comment{done compressing}

 \item \T $V' \gets \left\{~v~|~v \in V~ \wedge ~v = L[v] ~\right\}$
            \comment{collect only the root vertices}

 \item \T $E' \gets \left\{~( L[ u ], L[ v ] )~|~(u, v) \in E~ \wedge ~L[ u ] \neq L[ v ] ~\right\}$
           \lcomment{$E'$ contains only edges among roots,}

           \rcomment{and no duplicate edges and self loops}

 \item \T \func{Par-Randomized-CC-2}$(~V', ~E', ~L ~)$
           \lcomment{use the algorithm from Figure \ref{fi:pcc} to solve the problem once the}

           \rcomment{number of edges reduces to a sufficiently small number}

 \item \T \xparallel \xfor each $v \in V$ \xdo $L[ v ] \gets L[ L[ v ] ]$
              \comment{map the solution back to the current instance}

 \item \xendif 

 \vspace{0.2cm}

 \end{enumerate}

 }
 \end{minipage}
 \vspace{0.1cm}
 }


 \vspace{-0.2cm}
 \caption{Parallel connected components (CC) based on edge sampling.}
 \label{fi:pcc-sampling}
 \vspace{-0.3cm}
 \end{center}
 \end{minipage}
 \end{figure*}



\newpage


\vspace{2cm}

\section*{APPENDIX 1: Input/Output Format for Task 2}

 Your code must read from standard input and write to standard output.
%
 \begin{itemize}
%
\vspace{-0.1cm}
 \item[--] {\bf Input Format:} The first line of the input will contain two integers giving the number of vertices ($n$)
 and the number of edges ($m$), respectively. Each of the next $m$ lines will contain two integers $u$ and $v$ ($1 \leq u, v \leq n$)
denoting an undirected edge between vertices $u$ and $v$. The edges will be sorted in nondecreasing order of the first vertex.
%
\vspace{-0.1cm}
 \item[--] {\bf Output Format:} The first line of the output will contain an integer $r$
giving the number of connected components in the input graph. This will be followed
by the size of each of the $r$ connected components, each on a separate line, ordered from the
largest to the smallest. The size
of a connected component is an integer giving the number of vertices in that component.
%
\vspace{-0.1cm}
 \item[--] {\bf Sample Input/Output:} {\tt /work/01905/rezaul/CSE638/HW2/samples} on Lonestar.
%
\vspace{-0.1cm}
 \item[--] {\bf Test Input/Output (see Table \ref{test-input}):} {\tt /work/01905/rezaul/CSE638/HW2/turn-in} on Lonestar.
%
 \end{itemize}

\begin{table}[ht!]
\vspace{-8pt}
%\begin{tabular}{|p{1.1cm}|p{2.5cm}|p{.9cm}|p{.9cm}|p{.8cm}|} % centered columns (4 columns)
\begin{center}
\begin{tabular}{|l|l|r|r|} % centered columns (4 columns)
\hline                        %inserts double horizontal lines
%
\scriptsize Graph
 & \scriptsize Description
 &\scriptsize $n$ &\scriptsize $m$
% &\scriptsize D
\\ % inserts table
\hline                  % inserts single horizontal line
\scriptsize as-skitter  & \scriptsize Internet topology graph, from traceroutes run daily in 2005  & \scriptsize 1.7M& \scriptsize 11M\\
% & \scriptsize 53 \\ % inserting body of the table
\scriptsize ca-AstroPh &\scriptsize Collaboration network of Arxiv Astro Physics &\scriptsize 18.7K& \scriptsize 396K\\
% & \scriptsize 42 \\
\scriptsize com-amazon&\scriptsize Amazon product network &\scriptsize 334K	& \scriptsize 925K\\
% & \scriptsize 141 \\
\scriptsize com-dblp &\scriptsize DBLP collaboration network &\scriptsize 317K	
	& \scriptsize 1M\\
% & \scriptsize 14 \\     % [1ex] adds vertical space
\scriptsize com-friendster &\scriptsize Friendster online social network
 &\scriptsize 65M	
	& \scriptsize 1.8B\\
% & \scriptsize 11 \\     % [1ex] adds vertical space
\scriptsize com-lj &\scriptsize LiveJournal online social network &\scriptsize 4M	
	& \scriptsize 34M\\
% & \scriptsize 12 \\     % [1ex] adds vertical space
\scriptsize com-orkut &\scriptsize Orkut online social network &\scriptsize 3M	& \scriptsize 117M\\
% & \scriptsize 5 \\     % [1ex] adds vertical space
\scriptsize roadNet-CA &\scriptsize Road network of California &\scriptsize 2M	& \scriptsize 2.7M\\
% & \scriptsize 5 \\     % [1ex] adds vertical space
\scriptsize roadNet-PA &\scriptsize Road network of Pennsylvania &\scriptsize 1M	& \scriptsize 1.5M\\
% & \scriptsize 5 \\     % [1ex] adds vertical space
\scriptsize roadNet-TX &\scriptsize Road network of Texas &\scriptsize 1.4M	& \scriptsize 1.9M\\
% & \scriptsize 5 \\     % [1ex] adds vertical space
\hline %inserts single line
\end{tabular}
\caption{Input graphs with \#vertices ($n$) and \#edges ($m$). 
% Note that $1K = 10^3$, $1M = 10^6$ and $1B = 10^9$.
%The diameters in the table show the maximum diameter explored by the BFS rather than the actual diameter of these graphs.
}
\vspace{-8pt}
%\end{minipage}
\label{test-input} % is used to refer this table in the text%\vspace{-5pt}
\end{center}
\end{table}


\newpage
\section*{APPENDIX 2: What to Turn in}

 One compressed archive file (e.g., zip, tar.gz) containing the following items.
%
 \begin{itemize}
%
\vspace{-0.1cm}
 \item[--] Source code, makefiles and job scripts.
%
\vspace{-0.1cm}
 \item[--] A PDF document containing all answers and plots.
%
\vspace{-0.1cm}
 \item[--] Output generated for the input files in {\tt /work/01905/rezaul/CSE638/HW2/turn-in/} on Lonestar.
           If the name of the input file is {\tt xxxxx-in.txt} in task $2(j)$, please name the output files as {\tt xxxxx-2j-CC-1-out.txt},
           {\tt xxxxx-2j-CC-2-out.txt} and {\tt xxxxx-2j-CC-3-out.txt} for \func{Par-Randomized-CC},
           \func{Par-Randomized-CC-2} and \func{Par-Randomized-CC-3}, respectively. 
%
 \item[--] Output files generated by Cilkview.
%
 \end{itemize}


\section*{APPENDIX 3: Things to Remember}
%
 \begin{itemize}
%
\vspace{-0.1cm}
 \item[--] {\bf Please never run anything that takes more than a minute or uses multiple cores on TACC login nodes.}
          TACC policy strictly prohibits such usage. They reserve the right to suspend your account if you
          do so. All runs must be submitted as jobs to compute nodes (even when you use Cilkview or PAPI).
%
\vspace{-0.1cm}
 \item[--] Please store all data in your work folder (\$WORK), and not in your home folder (\$HOME).
%
\vspace{-0.1cm}
 \item[--] When measuring running times please exclude the time needed for reading the input and writing
          the output. Measure only the time needed by the algorithm. Do the same thing when you
          run Cilkview.
%
\vspace{-0.1cm}
 \item[--] Please make sure that speedup values for trial runs (or measured speedups) are
           included in the Cilkview plots you generate.
%
 \end{itemize}


\end{document}



